\documentclass[12pt,leqno]{amsart}
\pagestyle{plain}
\usepackage{latexsym,amsmath,amssymb}
%\usepackage[notref,notcite]{showkeys} 

\setlength{\oddsidemargin}{1pt}
\setlength{\evensidemargin}{1pt}
\setlength{\marginparwidth}{30pt} % these gain 53pt width
\setlength{\topmargin}{1pt}       % gains 26pt height
\setlength{\headheight}{1pt}      % gains 11pt height
\setlength{\headsep}{1pt}         % gains 24pt height
%\setlength{\footheight}{12 pt} 	  % cannot be changed as number must fit
\setlength{\footskip}{24pt}       % gains 6pt height
\setlength{\textheight}{650pt}    % 528 + 26 + 11 + 24 + 6 + 55 for luck
\setlength{\textwidth}{460pt}     % 360 + 53 + 47 for luck



\def\dsp{\def\baselinestretch{1.35}\large
\normalsize}
%%%%This makes a double spacing. Use this with 11pt style. If you
%%%%want to use this just insert \dsp after the \begin{document}
%%%%The correct baselinestretch for double spacing is 1.37. However
%%%%you can use different parameter.


\def\U{{\mathcal U}}









\begin{document}
 \bigskip
 \centerline{\bf{Orthogonal Projections}}

 \bigskip

 {\bf Defintion:} For $S$ a subspace of $\mathbb{R}^n$ with basis vectors $\{\mathbf{u}_1, \mathbf{u}_2, \dots, \mathbf{u}_k\}$ and a point $\mathbf{g} \in \mathbb{R}^n$ then $\mathbf{a} \in S$ is the orthogogonal projection of $\mathbf{g}$ onto $S$ if:
 $$\forall i\leq k\ \  \ \ (\mathbf{g} - \mathbf{a})\cdot\mathbf{u}_i = 0 $$
 \medskip
{\bf Proposition 1: } If an orthogonal projection of $\mathbf{g}$ onto $S$ exists it is unique
\begin{proof} Let $\mathbf{a}$ and $\mathbf{b}$ be orthogonal projections of $\mathbf{g}$ onto $S$ and consider:
$$\Rightarrow \mathbf{a} = a_1\mathbf{u}_1 + \dots + a_k\mathbf{u}_k  \text{  and  } \mathbf{b} = b_1\mathbf{u}_1 + \dots + b_k\mathbf{u}_k $$
Consider:
$$\mathbf{a} - \mathbf{b} = (a_1 - b_1)\mathbf{u}_1 + \dots + (a_k - b_k)\mathbf{u}_k$$
For every $i$ let $z_i := a_i - b_i$ and we can write:
$$\mathbf{a} - \mathbf{b} = z_1\mathbf{u}_1 + \dots + z_k\mathbf{u}_k$$
By defintion we have: 
$$0 = (\mathbf{g} - \mathbf{a})\cdot\mathbf{u}_k = (\mathbf{g} - \mathbf{a} + \mathbf{b} - \mathbf{b})\cdot\mathbf{u}_k = (\mathbf{g} - \mathbf{b} - (\mathbf{a} - \mathbf{b}) )\cdot\mathbf{u}_k $$
$$= (\mathbf{g} - \mathbf{b})\cdot\mathbf{u}_k - (\mathbf{a} -\mathbf{b})\cdot\mathbf{u}_k = - (\mathbf{a} -\mathbf{b})\cdot\mathbf{u}_k $$ 
$$\iff (\mathbf{a} -\mathbf{b})\cdot\mathbf{u}_k = 0$$
\end{proof}


\end{document}